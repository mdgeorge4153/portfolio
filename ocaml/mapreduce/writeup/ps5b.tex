\documentclass{pset}

\usepackage{tikz}
\usetikzlibrary{matrix}
\usetikzlibrary{calc}
\usetikzlibrary{positioning}
\usetikzlibrary{decorations}
\usetikzlibrary{decorations.pathreplacing}
\usetikzlibrary{arrows}
\usetikzlibrary{shapes.misc}

\psnum{5b}
\date{Due Thursday, April 17}
\versionnumber{3}

\newcommand{\sout}[1]{\tikz[baseline] \node[anchor=text,shape=cross out,draw=red] {#1};}

\begin{document}
\maketitle{}

\section*{Overview}

Many modern computational problems require processing very large data sets.
Google's MapReduce framework is a platform for distributing programs that
handle big data across a large number of computers.

In this problem set, you will implement a simplified version of the MapReduce
framework.  You will also implement various applications that make use of your
framework to process large data sets.

\section*{Objectives}

This assignment is designed to help you learn the following skills:
\begin{itemize}
\item Writing asynchronous programs in an event-driven style
\item Developing distributed systems
\item Working with the MapReduce paradigm
\item Developing software with a partner
\end{itemize}

This assignment also has a small component on writing formal proofs.

\section*{Recommended Reading}

\begin{itemize}
\item We have provided \link{http://www.cs.cornell.edu/Courses/cs3110/2014sp/hw/5/doc/index.html}{documentation}
      for a subset of the Async library that should be sufficient for this
      assignment.
      \begin{note}{Important}
      Async contains a very large number of libraries and functions.  While you
      are free to use anything from the full Async library, sticking to the
      subset we have documented will make the project simpler.
      \end{note}
\item Real World OCaml, \link{https://realworldocaml.org/v1/en/html/concurrent-programming-with-async.html}{chapter 18}
\item Lectures \link{http://www.cs.cornell.edu/courses/cs3110/2014sp/lectures/17/concurrency.html}{17} and
               \link{http://www.cs.cornell.edu/courses/cs3110/2014sp/lecture_notes.html}{18},
      recitation 
               \link{http://www.cs.cornell.edu/courses/cs3110/2014sp/lecture_notes.html}{17}.
\item \link{http://git-scm.com/book/en/Getting-Started}{Pro Git}, chapters 1 and 2
\end{itemize}

\newpage{}

\part{Software engineering}

This problem set is larger than previous problem sets.  To help you manage the
complexity, you will be required to work with a partner, make use of version
control software, and meet with a TA to discuss your approach.

\subsection*{Partners}

You are required to work with a partner for this problem set.  You and your
partner should meet early to jointly discuss your design and division of
responsibility.

Each partner is responsible for understanding all parts of the assignment, but
your choice of who writes what code is up to you.

\subsection*{Source control}

You must use a version control system to manage your development.  We
recommend \filename{git}, but you are free to use another system if you prefer.
Both \link{http://github.com/}{github} and
\link{http://bitbucket.com/}{bitbucket} provide free private git repositories
for educational purposes.

You should submit your source control logs.  We will be looking for small
self-contained commits with clear commit messages.  You can generate a git log
using the command
\begin{ocaml}
git log --stat > log.txt
\end{ocaml}

\subsection*{Problem set check-in}

You are required to attend a short meeting with your partner and a TA.  You
should come to the meeting prepared to discuss your approach to the assignment,
the division of labor between yourself and your partner, and any questions you
may have.

Meetings will be held during normal consulting hours during the week of the
7th.  You should sign up for a slot on CMS.

\newpage{}

\part{Async warmup}
\label{part:warmup}
These exercises are intended to get you used to asynchronous computation and
the Async programming environment.

\exercise{}

Write  a function \code{fork} that takes a deferred and two blocking functions,
and runs the two functions concurrently when the deferred becomes determined:

\begin{ocaml}
val fork : 'a Deferred.t -> ('a -> 'b Deferred.t)
                         -> ('a -> 'c Deferred.t) -> unit
\end{ocaml}

The output of the two blocking functions should be ignored.

\exercise{}
Using \emph{only} \code{(>>=)}, \code{return}, and the ordinary \code{List}
module functions, implement a function with the following specification:
\begin{ocaml}
val deferred_map : 'a list -> ('a -> 'b Deferred.t) -> 'b list Deferred.t
\end{ocaml}
This function should take a list \code{l} and a blocking function \code{f}, and
should apply \code{f} \emph{concurrently} to each element of \code{l}.  When
all of the calls to \code{f} are complete, \code{deferred_map} should return a
list containing all of their values.

\begin{note}{Update (version 2)}
\label{update:deferred_map}
In fact, this is \emph{not} the same as the specification for Deferred.List.map
(in fact the Jane Street documentation does not give a specification for
\code{Deferred.List.map}).

\code{Deferred.List.map [1;2] f} will not begin running \code{f 2} until the 
Deferred returned by \code{f 1} is determined (informally, until \code{f 1}
completes).  If you want use the function specified above for your
implementation, either use your implementation of \code{deferred_map} or supply
the optional argument \code{~how:`Parallel} to \code{Deferred.List.map}:
\begin{ocaml}
Deferred.List.map ~how:`Parallel [1;2] f
\end{ocaml}
\end{note}

\exercise{}

Implement the asynchronous queue interface defined in \filename{aQueue.mli}.
You may find the \code{Async.Std.Pipe} module useful for this exercise.

% \newpage{}
% \exercise{}
% 
% Write a function \code{backup_plan} that takes a list of blocking functions and
% tries them one at a time, with a 5 second delay in between.  It should return
% a \code{Deferred.t} that becomes determined when the first of the jobs
% finishes.  Once any of the jobs have been successfully completed, no new jobs
% should be started.
% 
% For example, suppose that we have jobs \code{j1}, \code{j2}, \code{j3}, and
% \code{j4}.  Suppose that \code{j1} takes 11 seconds.  \code{j2} takes 9 seconds,
% \code{j3} takes 7 seconds, and \code{j4} takes 25 seconds.  Calling
% \code{backup_plan [j1; j2; j3; j4]} would initiate the following sequence of
% events:
% 
% \begin{tikzpicture}[
%   y=-0.25cm,
%   timeline/.style={[-]}
% ]
% 
% \matrix (labels) at (1,-1) [matrix of nodes,anchor=north west,nodes={anchor=base west}] {
%   |(js1)|0:00: $J_1$ starts \\
%   |(js2)|0:05: $J_2$ starts \\
%   |(js3)|0:10: $J_3$ starts \\
%   |(jf2)|0:11: $J_2$ finishes with result $R_2$ \\
%   |(ret)|0:11: the deferred returned by \texttt{backup\_plan} is determined with value $R_2$ \\
%   |(jf1)|0:14: $J_1$ finishes with result $R_1$ \\
%   |(jf3)|0:17: $J_3$ finishes with result $R_3$ \\
%   |(js4)|      $J_4$ is never run, because a result has been computed before it is scheduled \\
% };
% 
% \coordinate (t-js1) at (0,0);
% \coordinate (t-js2) at (0,5);
% \coordinate (t-js3) at (0,10);
% \coordinate (t-jf2) at (0,11);
% \coordinate (t-ret) at (0,11);
% \coordinate (t-jf1) at (0,14);
% \coordinate (t-jf3) at (0,17);
% \coordinate (t-js4) at (0,15);
% 
% \draw[->] (0,-1) node[above] {time} -- (0,20);
% 
% \coordinate (tl) at (-.1,0);
% \coordinate (tr) at (+.1,0);
% 
% \foreach \i in {js1,js2,js3,jf2,ret,jf1,jf3}
%   \draw (\i.west) -- (tr |- t-\i) -- (tl |- t-\i);
% 
% \draw[dotted] (js4.west) -- (tr |- t-js4) -- (tl |- t-js4);
% 
% \foreach \j in {1,2,3} {
%   \draw[dotted]   (0,0   |- t-js\j) -- (-\j,0 |- t-js\j);
%   \draw[dotted]   (0,0   |- t-jf\j) -- (-\j,0 |- t-jf\j);
%   \draw[timeline] (-\j,0 |- t-js\j) -- node[near start,fill=white,inner sep=2pt] {$J_\j$} (-\j,0 |- t-jf\j);
% }
% 
% %\draw[->] (time |- 0,-.1) -- (time |- 0,2);
% \end{tikzpicture}
% 
% You may find the \code{IVar} module useful for this exercise.
% 
\newpage{}
\part{MapReduce Framework}

Modern applications rely on the ability to manipulate massive data sets in an
efficient manner.  One technique for handling large data sets is to distribute
storage and computation across many computers.  Google's MapReduce is a
computational framework that applies functional programming techniques to
parallelize applications.

\subsection*{MapReduce Overview}

MapReduce was spawned from the observation that a wide variety of applications
can be structured into a \emph{map phase}, which transforms independent data
points, and a \emph{reduce phase} which combines the transformed data in a
meaningful way.  This is a very natural generalization of folding.

MapReduce jobs provide the code to run in these two phases by implementing the
\code{MapReduce.Job} interface.  A MapReduce Job is executed as follows:
\begin{itemize}
\item The \code{map} function takes a single \code{input} and transforms the
      value into a collection of intermediate key, value pairs.  The \code{map}
      function is called once for each element of the input list.
      \begin{ocaml}
      val map : input -> (key * inter) list Deferred.t
      \end{ocaml}

\item The \code{map} results are then combined --- values associated with the
      same key are merged into a single list.

\item Finally, the \code{reduce} function takes a \code{key} and an \code{inter list}
      to compute the output corresponding to that key.
      \code{reduce} is called once for each independent key.
      \begin{ocaml}
      val reduce : key * inter list -> output Deferred.t
      \end{ocaml}
\end{itemize}

The advantage of this design is that each call to \code{map} or \code{reduce}
is independent, so the work can be distributed across a large number of
machines.  This allows MapReduce applications to process very large data sets
quickly while using limited resources on each individual machine.

\subsection*{An example: Word Count}

Figure~\ref{fig:wordcount} shows a distributed execution of a word counting
application.  The input to the application is a list of filenames.  During the
map phase, the controller sends a \code{MapRequest} message for each input
filename to a mapper.  The mapper invokes the \code{map} function in the
\code{WordCount.Job} module, which reads in the corresponding file and produces
a list of \code{(word, count)} pairs.  The mapper then sends these pairs back to
the controller.

Once the controller has collected all of the intermediate \code{(word,count)}
pairs, it groups all of the pairs having the same word into a single list, and
sends a \code{ReduceRequest} to a reducer.  The reducer invokes the
\code{reduce} function of the \code{WordCount.Job} module, which simply returns
the sum of all of the counts.  The reducer sends this output back to the
controller, which collects all of the reduced outputs and returns them to the
\code{WordCount} application.

\begin{figure}
\begin{center}
\begin{tikzpicture}[
  worker/.style = {draw,rounded corners},
  controller/.style = {draw},
]
\node [controller] (c1) at (0,8) {Controller};
\matrix[column sep=1em, matrix of math nodes, nodes={worker}]
  (mappers) at (0,4)
{
  M_1 & M_2 & |[draw=none]| \cdots & M_3 & M_4 \\
};
\node [controller] (c2) at (0,0)  {Controller};
\matrix[column sep=1em, matrix of math nodes, nodes={worker}]
  (reducers) at (0,-4)
{
  R_1 & R_2 & |[draw=none]| \cdots & R_3 & R_4 \\
};
\node [controller] (c3) at (0,-8)  {Controller};

\foreach \i in {1,2,4,5} {
  \draw[->] (c1) .. controls +(0,-1) and +(0,1) .. (mappers-1-\i);
  \draw[->] (mappers-1-\i) .. controls +(0,-1) and +(0,1) .. (c2);
  \draw[->] (c2) .. controls +(0,-1) and +(0,1) .. (reducers-1-\i);
  \draw[->] (reducers-1-\i) .. controls +(0,-1) and +(0,1) .. (c3);
}

%     badger badger badger badger     badger badger badger badger
%     badger badger badger badger     mushroom mushroom!
%     badger badger badger badger     badger badger badger badger
%     badger badger badger badger     mushroom mushroom!
%     snaaake snaaake, oh, it's a snaaake!
\node at (0,9) {\begin{minipage}{5in}
\begin{ocaml}
RemoteController(WordCount.Job).map_reduce
  ["mushroom soup"; "badger badger $\cdots$"; $\cdots$];;
\end{ocaml}
\end{minipage}
};

\node at (0,-9.5) {\begin{minipage}{5in}
\begin{ocaml}
return [("mushroom",3); ("badger",12);
        ("soup",1);     ("snake!",7); $\cdots$]
\end{ocaml}
\end{minipage}
};

\node [anchor=south east] at (-2,5.5) {\begin{minipage}{1.8in}
\begin{ocaml}
MapRequest
  "mushroom soup"
\end{ocaml}
\end{minipage}
};

\node [anchor=north east] at (-2,2.5) {\begin{minipage}{1.8in}
\begin{ocaml}
MapResult [
  ("mushroom",1);
  ("soup",1)
]
\end{ocaml}
\end{minipage}
};

\node [anchor=south west] at (2,5.5) {\begin{minipage}{1.8in}
\begin{ocaml}
MapRequest
  "badger badger $\cdots$"
\end{ocaml}
\end{minipage}};

\node [anchor=north west] at (2,2.5) {\begin{minipage}{1.8in}
\begin{ocaml}
MapResult [
  ("badger", 12);
  ("mushroom", 2);
  ("snake!", 3);
]
\end{ocaml}
\end{minipage}};

\node [anchor=south east] at (-2,-2.5) {\begin{minipage}{1.8in}
\begin{ocaml}
ReduceRequest
  ("badger",[12])
\end{ocaml}
\end{minipage}
};

\node [anchor=south west] at (2,-2.5) {\begin{minipage}{1.8in}
\begin{ocaml}
ReduceRequest
  ("mushroom",[1;2])
\end{ocaml}
\end{minipage}
};



\node [anchor=north east] at (-2,-5.5) {\begin{minipage}{1.8in}
\begin{ocaml}
ReduceResult 12
\end{ocaml}
\end{minipage}
};

\node [anchor=north west] at (2,-5.5) {\begin{minipage}{1.8in}
\begin{ocaml}
ReduceResult 3
\end{ocaml}
\end{minipage}
};

\draw [decorate,decoration={brace,amplitude=1em}] (7,8) -- node[sloped,above=1em] {map phase} (7,0);
\draw [decorate,decoration={brace,amplitude=1em}] (7,-.1) -- node[sloped,above=1em] {reduce phase} (7,-8);

\end{tikzpicture}
\end{center}
\caption{Execution of a WordCount MapReduce job}
\label{fig:wordcount}
\end{figure}

\subsection*{Communication protocol}

In our implementation of MapReduce, the Controller communicates with Workers
by sending and receiving the messsages defined in the \code{Protocol} module.

Messages are strongly typed; a message from the Controller to the Worker will
have type \code{WorkerRequest.t}; responses have the type
\code{WorkerResponse.t}.  Messages can be sent and received by using the
\code{send} and \code{receive} functions of the corresponding module.  For
example, the Controller should call \code{WorkerRequest.send} to send a request
to a worker; the worker will call \code{WorkerRequest.receive} to receive it.

The \code{WorkerRequest} and \code{WorkerResponse} modules are parameterized on
the \code{Job}, so that the messages can contain data of the types defined by
the \code{Job}.  This means that before the worker can call \code{receive}, it
needs to know which \code{Job} it is running.  As soon as the controller
establishes a connection to a worker, it should send a single line containing
the name of the job.  After the job name is sent, the controller should only
send \code{WorkerRequest.t}s and receive \code{WorkerResponse.t}s.

We have provided code in the \code{Worker} module's \code{init} function that
receives the job name and calls \code{Worker.Make} with the corresponding
module.

Once a connection is established and the job name is sent, the Controller will
send some number of \code{WorkerRequest.MapRequest} and
\code{WorkerRequest.ReduceRequest} messages to the worker.  The worker will
process these messages and send \code{WorkerRequest.MapResult} and
\code{WorkerRequest.ReduceResult} messages respectively.  When the job is
complete, the controller should close the connection.

\begin{note}{Update (version 2)}
\label{update:shutdown}
We have added the function \code{Async.Std.Socket.shutdown} to our
documentation, which you can call to close the connection.
\end{note}

\subsection*{Error handling}

There are a variety of errors that you will have to consider.
\begin{description}
\item[infrastructure failure]
If the controller is unable to connect to a worker, or if a connection is broken
while it is in use, or if the worker misbehaves by sending an inappropriate
message, the controller should simply close the connection to the worker and
continue processing the job using the remaining workers.

If all of the workers die, the \code{map_reduce} function should raise an
\code{InfrastructureFailure} exception.

If a worker encounters an error when communicating with the controller, it
should simply close the connection.

\item[application failure]
If the application raises an exception while executing the \code{map} or
\code{reduce} functions, then the worker should return a \code{JobFailed}
message.  Upon receiving this message, the controller should cause
\code{map_reduce} to raise a \code{MapFailed} or \code{ReduceFailed} exception.

The name and stack trace of the original exception can be found using the
\code{Printexc} module from the OCaml standard library; these should be returned
to the controller in the \code{JobFailed} message, and used to construct the
\code{MapFailed} exception.

\begin{note}{Update (version 2)}
\label{update:exceptions}
We did not define the \code{InfrastructureFailure}, \code{MapFailed} or
\code{ReduceFailed} exceptions in the release code.  Feel free to define them
yourself or to raise any other exception (for example by calling
\code{failwith}.
\end{note}

\end{description}

\exercise{Implement RemoteController}

Implement the \code{RemoteController} module.  The \code{init} function should
simply record the provided list of addresses for future invocations of
\code{Make.run}.

The \code{Make.map_reduce} function is responsible for executing the MapReduce
job.  It should use \code{Tcp.connect} to connect to each of the workers that
were provided during \code{init}.  It should then follow the protocol described
above to complete the given \code{Job}.

You can use the controller to run a given app by running the
\filename{controllerMain.ml}:
\begin{ocaml}
% cs3110 compile controllerMain.ml
% cs3110 run controllerMain.ml <app_name> <app_args>
\end{ocaml}

\exercise{Implement Worker}

Implement the \code{Worker.Make} module in the \filename{map\_reduce}
directory.  The \code{Make.run} function should receive messages on the
provided \code{Reader.t} and respond to them on the provided \code{Writer.t}
according to the protocol described above.

You can run the worker on a given port by invoking \filename{workerMain.ml}:
\begin{ocaml}
% cs3110 compile workerMain.ml
% cs3110 run workerMain.ml 31100
\end{ocaml}

The list of addresses and ports that the controller will try to connect to is
given in the file \filename{addresses.txt}.

\exercise{[Karma] handle slow workers}

It is possible that some workers may simply be slow.  As an optional Karma
problem, you may detect whether a worker is taking more than a few seconds to
process a job, and add a second worker to process the same job if it is.  If
both workers are taking too long, you can add a third worker, and so on.

The old workers should not be terminated; the first of these workers to respond
should determine the output of the job.  You may find \code{Ivar}s useful for
this task.

\newpage{}
\part{MapReduce Applications}

In this part of the assignment you will implement various MapReduce
applications.

A MapReduce application implements the \code{MapReduce.App} interface, which
provides a \code{main} function.  This function will typically read some input,
and then invoke the provided controller with one or more \code{Job}s.
We have provided you with the \code{WordCount} example application described
above to get you started.

We have also provided you with a local controller that you can use to test your
apps without a completed implementation of the MapReduce framework.  To run an
app locally, simply pass the \filename{-local} option to
\filename{controllerMain.ml}.  You should be able to run \code{WordCount} out
of the box:
\begin{ocaml}
% cs3110 compile controllerMain.ml
% cs3110 run controllerMain.ml -local wc ../writeup/ps5b.tex
\end{ocaml}

\exercise{Inverted Index}

An inverted index is a mapping from words to the documents in which they
appear.  Complete the \code{InvertedIndex} module (in \filename{apps/index})
that takes in a master list of files, and computes an index on those files.

For example, if the files \filename{master.txt}, \filename{zar.txt} and
\filename{doz.txt} contained the following:

\begin{center}
\begin{tikzpicture}
\node (master) {\begin{minipage}{1in}
\begin{ocaml}
zar.txt
doz.txt
\end{ocaml}
\end{minipage}};
\node[anchor=base west] at (master.north west) {\filename{master.txt}};

\node[right=1cm of master] (zar) {\begin{minipage}{2.1in}
\begin{ocaml}
ocaml is fun
fun fun fun
\end{ocaml}
\end{minipage}
};
\node[anchor=base west] at (zar.north west) {\filename{zar.txt}};

\node[right=1cm of zar] (doz) {\begin{minipage}{2.1in}
\begin{ocaml}
because fun
is a keyword
\end{ocaml}
\end{minipage}
};
\node[anchor=base west] at (doz.north west) {\filename{doz.txt}};
\end{tikzpicture}
\end{center}

then running the \filename{index} app on \filename{master.txt} should produce
the output
\begin{ocaml}
[("ocaml", ["zar.txt"]);            ("is", ["zar.txt"; "doz.txt"]);
 ("fun",   ["zar.txt"; "doz.txt"]); ("because", ["doz.txt"]);
 ("a",     ["doz.txt"]);            ("keyword", ["doz.txt"])]
\end{ocaml}

\exercise{Genetic Sequence Alignment}

The goal of this exercise is to identify which fragments of a DNA sequence
appear within a longer reference sequence.  This is an important problem from
the field of computational biology.

A DNA sequence is a string made from the letters G, C, A and T.  The input to
your application will be a single long sequence (called the ``reference'') and
a collection of short sequences (called ``reads'').  Your goal is to identify
subsequences of the reads that match some part of the reference.

For example, if you are given the reference sequence
\begin{ocaml}
ref:  GATCTCTATGCAAAATACGTATTTGTACGTCCACCCTCGGAGTGGTG
\end{ocaml}
and one of the reads is
\begin{ocaml}
read: CGTATTTGTACATCCACCCTCGG
\end{ocaml}
your goal is to discover that they match well when lined up as follows:
\begin{ocaml}
match:                 ___________ ___________
ref:  GATCTCTATGCAAAATACGTATTTGTACGTCCACCCTCGGAGTGGTG
read:                  CGTATTTGTACATCCACCCTCGG
\end{ocaml}

This problem can be solved using two MapReduce jobs as follows. \begin{itemize}

\item In the first map phase, the input reference and reads will be broken into
10 character sequences (called ``10-mers,'' or more generally ``$k$-mers'').
For example, the read \code{"AGCTAGCTCAGTACC"} would be mapped as follows:

\begin{ocaml}
read X: AGCTAGCTCAGTACC
output: AGCTAGCTCA       occurs in read X at offset 0
         GCTAGCTCAG      occurs in read X at offset 1
          CTAGCTCAGT     occurs in read X at offset 2
           TAGCTCAGTA    occurs in read X at offset 3
            AGCTCAGTAC   occurs in read X at offset 4
             GCTCAGTACC  occurs in read X at offset 5
\end{ocaml}

The mapper will output the $k$-mers as keys, and identifying information (such
as the source sequence and offset) as values.

\item The first reduce phase collects all of the occurrences of the given
$k$-mer, and outputs a list of all possible matches between a sequence and a
read.

\item The second map phase will take the $k$-mer matches as input and will
output them with keys given by the identities of the reference and read to which
they correspond.

\item The final reduce phase will combine adjacent shared $k$-mers.  For
example, if the $10$-mers at offsets $5$, $6$, and $7$ of a read $r$ match the
reference at offsets $17$, $18$, and $19$, then we know that the subsequence of
$r$ of length $12$ starting at offset $5$ matches the reference at offset
\sout{$15$} $17$.

\label{update:dna}
\end{itemize}

At the end of these two phases, the output will be a list of partial matches
between reads and references.

We have provided you with starter code to load files containing the data.  You
must implement the \code{DnaSequencing.App.run} function which takes a list of
reads and references and outputs a list of matches between them.

\newpage{}
\section*{Getting started}

To help you get oriented, here is a brief summary of the files in the release:
\begin{itemize}
\item The \filename{map\_reduce} folder contains all of the infrastructure code \begin{itemize}
    \item The \code{MapReduce} module defines the interface between the apps
          and the infrastructure.
    \item \code{RemoteController} and \code{Worker} are the modules that you
          have to implement.
    \item \code{LocalController} contains a working non-distributed
          implementation of the \code{Controller} interface.  It makes
          use of \code{Combiner} for the combine phase.
    \end{itemize}
\item The \filename{apps} folder contains the apps and the utilities that they
      use.
\item The \filename{async} folder contains the stubs for the async warmup
      exercises in part~\ref{part:warmup}.
\end{itemize}

\section*{Comments}

In addition to the usual comments, we are particularly interested in your
feedback about using Async for this project.

\section*{Karma suggestions}

There are a lot of fun things you can do with MapReduce.  For example: \begin{itemize}
\item There are a very large number of applications that can be implemented in
      the MapReduce framework.  Find one that interests you and code it up!

\item Brute force algorithms can work well when you have lots of brutes.  You
      could adapt your solver from PS3 to run as a MapReduce app.

\item You could think about how the software design would change if you wanted
      to make the workers themselves concurrent (and write up a summary)

\item You could also think about how to extend the design so that the mappers
      communicate directly with the reducers to reduce the load on the
      controller.
\end{itemize}

\end{document}
