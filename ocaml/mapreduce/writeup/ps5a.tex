\documentclass{pset}

\usepackage{bussproofs}

\usepackage{tikz}
\usetikzlibrary{matrix}
\usetikzlibrary{calc}
\usetikzlibrary{decorations}
\usetikzlibrary{decorations.pathreplacing}
\usetikzlibrary{arrows}

\newcommand{\TODO}[1][fix]{\textbf{[TODO --- #1]}}

\psnum{5a}
\date{Thursday, April 17}
\versionnumber{2}

\begin{document}
\maketitle{}

\section*{Overview}

In this assignment you will explore using logic to express and prove
formal properties.

\section*{Objectives}

This short assignment is designed to help you learn the following
skills:
\begin{itemize}
\item Expressing intuitive properties using formal logic
\item Understanding partial correctness specifications
\item Developing proofs using natural deduction
\end{itemize}

\section*{Recommended Reading}

\begin{itemize}
\item Lectures 14, 15, and 16
\item Recitations 14 and 15
\end{itemize}

\section*{What to turn in}
You should submit your solutions in a single file
\filename{logic.pdf}. Any comments you wish to make can go in
\filename{comments.txt} or \filename{comments.pdf}. If you choose to
submit any Karma work, you may submit the file \filename{karma.pdf}
(be sure to describe what you've done in the comments file).

\newpage{}

\exercise{}

Construct a proof tree for each of the following formulas:

\begin{enumerate}[(a)]
\item $P \wedge Q \Longrightarrow Q \wedge P$
\item $(P \wedge Q \Longrightarrow R) \Longrightarrow (P \Longrightarrow (Q \Longrightarrow R))$
\item $(P \vee Q \Longrightarrow R) \Longrightarrow (P \Longrightarrow R) \wedge (Q \Longrightarrow R)$
\item $\exists x.\, \neg P(x) \Longrightarrow \neg(\forall x.\, P(x))$
\end{enumerate}

\vfill{}

\exercise{}

Consider the following OCaml code:

\begin{ocaml}
(* pre: ??? 
 * post: nth [v0;...;vi;...;vn] i returns vi 
*)
let rec nth (l:'a list) (i:int) = 
  match i,l with 
   | _,[] -> 
     failwith "empty list"
   | i,_ when i < 0 -> 
     failwith "negative index"
   | 0,h::_ -> 
     h
   | _,_::t -> 
     nth t (i - 1)
\end{ocaml}

\begin{enumerate}[(a)]
\item Write a precondition that, when combined with the stated
  postcondition, yields a valid partial correctness specification for
  \code{nth}. For full credit, your precondition should impose as few
  constraints on the input as possible.

\item Define a mapping $f$ from \code{nth}'s inputs \code{(l,i)} to
  the natural numbers and briefly argue that this number decreases on
  each recursive call.
\end{enumerate}

\vfill{}

\newpage 

\exercise{} The following proofs are all incorrect. For each proof,
give the names of valid instances of axioms and inference rules and
write ``Bogus'' for invalid instances of axioms and inferences
rules. For example, given the following (non)-proof, 
%
\begin{prooftree}
\AxiomC{}
\RightLabel{(1)}
\UnaryInfC{$P \vdash P$}
\AxiomC{}
\RightLabel{(2)}
\UnaryInfC{$P\vdash Q$}
\RightLabel{(3)}
\BinaryInfC{$P \vdash P \wedge Q$}
\RightLabel{(4)}
\UnaryInfC{$\vdash P \Longrightarrow P \wedge Q$}
\end{prooftree}
%
you would write:
%
\begin{enumerate}[(1)]
\item Assumption
\item Bogus
\item $\wedge$-introduction
\item $\Longrightarrow$-introduction
\end{enumerate}

\begin{enumerate}[(a)]
\item 
\begin{prooftree}
\AxiomC{}
\RightLabel{(1)}
\UnaryInfC{$P \vee Q \vdash P$}
\AxiomC{}
\RightLabel{(2)}
\UnaryInfC{$P \vee Q \vdash Q$}
\RightLabel{(3)}
\BinaryInfC{$P \vee Q \vdash P \wedge Q$}
\RightLabel{(4)}
\UnaryInfC{$\vdash P \vee Q \Longrightarrow P \wedge Q$}
\end{prooftree}

\vfill{}

\item 
\begin{prooftree}
\AxiomC{}
\RightLabel{(1)}
\UnaryInfC{$P \vdash P$}
\AxiomC{}
\RightLabel{(2)}
\UnaryInfC{$P \vdash P$}
\RightLabel{(3)}
\AxiomC{}
\UnaryInfC{$P \vdash P \Longrightarrow \bot$}
\RightLabel{(4)}
\BinaryInfC{$P \vdash \bot$}
\RightLabel{(5)}
\UnaryInfC{$P \vdash \neg P$}
\RightLabel{(6)}
\BinaryInfC{$\vdash P \Longrightarrow P \wedge \neg P$}
\end{prooftree}

\vfill{}

\item 
\begin{prooftree}
\AxiomC{}
\RightLabel{(1)}
\UnaryInfC{$\exists x.\, P(x) \vdash \exists x.\, P(x)$}
\AxiomC{}
\RightLabel{(2)}
\UnaryInfC{$\exists x.\, P(x), P(a) \vdash P(a)$}
\RightLabel{(3)}
\UnaryInfC{$\exists x.\, P(x), P(a) \vdash \forall x.\, P(x)$}
\RightLabel{(4)}
\UnaryInfC{$\exists x.\, P(x) \vdash P(a) \Longrightarrow \forall x.\, P(x)$}
\RightLabel{(5)}
\BinaryInfC{$\exists x.\, P(x)\vdash \forall x.\, P(x)$}
\RightLabel{(6)}
\UnaryInfC{$\vdash \exists x.\, P(x) \Longrightarrow \forall x.\, P(x)$}
\end{prooftree}

\end{enumerate}

\vfill{}

\newpage
\section*{Karma suggestions}

\begin{itemize}
\item Prove $(P \Longrightarrow (Q \Longrightarrow R)) \Longrightarrow (P \wedge Q \Longrightarrow R)$

\item Prove $(P \Longrightarrow R) \wedge (Q \Longrightarrow R) \Longrightarrow (P \vee Q \Longrightarrow R)$

\item Prove that your precondition for  \code{nth} yields a valid partial correctness specification.

\item Prove that your precondition for  \code{nth} yields a valid total correctness specification.

\end{itemize}

\section*{Comments}

We would like to know how this assignment went for you.  Were there
any parts that you didn't finish or wish you had done in a better way?
Which parts were particularly fun or interesting?  Did you do any
Karma problems?

\end{document}

